\section{Introduction}
The field of mathematical optimisation faces a problem in that the vast majority of available software packages are designed in a manner that disallows easy modification or adaption of the software internals. This is due to these packages using compiled languages that do not lend themselves to easy modification, as well as being highly optimised for performance. While these are not problems for the industrial use of this software, it can be problematic in an academic setting where the ability to experiment with aspects of an algorithm is generally more desirable than the performance of the software. This project intends to tackle this by implementing several optimisation software packages for the Julia programming language which are designed to provide a powerful API for implementing new algorithms, as well as making it simple to modify and experiment with existing algorithms. Julia is a uniquely designed language that allows for this goal to be approached in a very different way than would be possible in more traditional languages, allowing for a simple design while retaining high performance. 

Julia has recently begun gaining traction in the numerical programming community due to its combination of powerful syntax, performance, and utility. It is commonly compared to the languages MATLAB, Python, and C, combining aspects of all three to create a very interesting new platform for programming in this space.

MATLAB is first and foremost designed as a mathematical language, limiting its utility in general purpose use-cases (as well as being limited by licensing). Python is famous for its speed of development and expressive syntax, but has slow performance and few restrictions on types, leading to hidden bugs. Finally, C remains a very high performance language, but struggles to provide an environment that allows for rapid implementation and extension of algorithms created with it. Julia takes each of these problems and provides a language with an expressive syntax focused on mathematical programming, a very powerful typing system, and performance that can near that of native C code in many circumstances. This flexibility has led to Julia being described as a solution to the `two language problem' by its authors, allowing both algorithm development and release software to be the same code-base \cite{TheBottomLine2018Julia:Problem}.

In this project we intend to utilise Julia to develop a set of native optimisation packages that are specifically designed to allow for extension and modification as needed. Contributions to the packages should be possible without requiring a detailed knowledge of the software architecture and the `tricks' needed for gaining high performance. The initial project plan identified a set of algorithms that are not currently available as native Julia code. The main aim of the project is to implement these algorithms within a structured framework that allows for similar algorithms to be added to the package easily. Each of the selected algorithms is representative of a different kind of solver (for example, row action methods, or direct search methods, discussed further in Section \ref{sec_background}), meaning that the collection of packages produced by the project will allow for simple implementation of a wide range of future algorithms. 

%Julia has the ability to interface with a large number of existing solvers (with over 30 contained in the JuliaOpt organisation alone \cite{JuliaOpt:Language}), but has comparatively few native solvers. This is an obvious area to try and move towards using native code. Through this project a groundwork will be laid to allow the easy implementation of powerful optimisation algorithms natively in Julia. The code will be easy to maintain, and easy to use as a reference for future work.
